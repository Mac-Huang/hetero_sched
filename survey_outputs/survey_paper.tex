
\documentclass[11pt,twocolumn]{article}
\usepackage[utf8]{inputenc}
\usepackage{amsmath,amsfonts,amssymb}
\usepackage{graphicx}
\usepackage{hyperref}
\usepackage{cite}
\usepackage{geometry}
\geometry{margin=1in}

\title{Deep Reinforcement Learning for Heterogeneous System Scheduling: A Comprehensive Survey}
\author{HeteroSched Research Team}
\date{\today}

\begin{document}
\maketitle

\begin{abstract}
The intersection of reinforcement learning (RL) and heterogeneous system scheduling represents a rapidly evolving research area with significant practical implications for modern computing infrastructure. This comprehensive survey examines the theoretical foundations, methodological approaches, and practical applications of RL techniques in heterogeneous scheduling environments.
\end{abstract}

\section{Introduction}
Modern computing systems increasingly rely on heterogeneous architectures that combine diverse processing units including CPUs, GPUs, FPGAs, and specialized accelerators.

\section{Background and Preliminaries}
This section establishes the foundational concepts necessary for understanding RL applications in heterogeneous scheduling.

\section{Theoretical Foundations}
The theoretical foundation of RL for heterogeneous scheduling rests on establishing convergence guarantees for multi-objective optimization in dynamic environments.

\section{Methodological Approaches}
Single-agent and multi-agent RL methods have been extensively developed for scheduling applications.

\section{Applications and Case Studies}
Real-world applications demonstrate the practical value of RL-based scheduling approaches.

\section{Current Challenges and Limitations}
Several challenges remain in scaling RL to production scheduling systems.

\section{Future Research Directions}
Emerging paradigms and methodological advances point toward exciting future opportunities.

\section{Conclusion}
This comprehensive survey reveals a rapidly maturing field with significant theoretical depth and practical impact.

\bibliographystyle{IEEEtran}
\bibliography{bibliography}

\end{document}
